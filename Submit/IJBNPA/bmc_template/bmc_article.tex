%% BioMed_Central_Tex_Template_v1.06
%%                                      %
%  bmc_article.tex            ver: 1.06 %
%                                       %

%%IMPORTANT: do not delete the first line of this template
%%It must be present to enable the BMC Submission system to
%%recognise this template!!

%%%%%%%%%%%%%%%%%%%%%%%%%%%%%%%%%%%%%%%%%
%%                                     %%
%%  LaTeX template for BioMed Central  %%
%%     journal article submissions     %%
%%                                     %%
%%          <8 June 2012>              %%
%%                                     %%
%%                                     %%
%%%%%%%%%%%%%%%%%%%%%%%%%%%%%%%%%%%%%%%%%

%%%%%%%%%%%%%%%%%%%%%%%%%%%%%%%%%%%%%%%%%%%%%%%%%%%%%%%%%%%%%%%%%%%%%
%%                                                                 %%
%% For instructions on how to fill out this Tex template           %%
%% document please refer to Readme.html and the instructions for   %%
%% authors page on the biomed central website                      %%
%% https://www.biomedcentral.com/getpublished                      %%
%%                                                                 %%
%% Please do not use \input{...} to include other tex files.       %%
%% Submit your LaTeX manuscript as one .tex document.              %%
%%                                                                 %%
%% All additional figures and files should be attached             %%
%% separately and not embedded in the \TeX\ document itself.       %%
%%                                                                 %%
%% BioMed Central currently use the MikTex distribution of         %%
%% TeX for Windows) of TeX and LaTeX.  This is available from      %%
%% https://miktex.org/                                             %%
%%                                                                 %%
%%%%%%%%%%%%%%%%%%%%%%%%%%%%%%%%%%%%%%%%%%%%%%%%%%%%%%%%%%%%%%%%%%%%%

%%% additional documentclass options:
%  [doublespacing]
%  [linenumbers]   - put the line numbers on margins

%%% loading packages, author definitions

%\documentclass[twocolumn]{bmcart}% uncomment this for twocolumn layout and comment line below
\documentclass{bmcart}

%%% Load packages
\usepackage{amsthm,amsmath}
%\RequirePackage[numbers]{natbib}
%\RequirePackage[authoryear]{natbib}% uncomment this for author-year bibliography
%\RequirePackage{hyperref}
\usepackage[utf8]{inputenc} %unicode support
%\usepackage[applemac]{inputenc} %applemac support if unicode package fails
%\usepackage[latin1]{inputenc} %UNIX support if unicode package fails

%%%%%%%%%%%%%%%%%%%%%%%%%%%%%%%%%%%%%%%%%%%%%%%%%
%%                                             %%
%%  If you wish to display your graphics for   %%
%%  your own use using includegraphic or       %%
%%  includegraphics, then comment out the      %%
%%  following two lines of code.               %%
%%  NB: These line *must* be included when     %%
%%  submitting to BMC.                         %%
%%  All figure files must be submitted as      %%
%%  separate graphics through the BMC          %%
%%  submission process, not included in the    %%
%%  submitted article.                         %%
%%                                             %%
%%%%%%%%%%%%%%%%%%%%%%%%%%%%%%%%%%%%%%%%%%%%%%%%%

\def\includegraphic{}
\def\includegraphics{}

%%% Put your definitions there:
\startlocaldefs
\endlocaldefs

%%% Begin ...
\begin{document}

%%% Start of article front matter
\begin{frontmatter}

\begin{fmbox}
\dochead{Research}

%%%%%%%%%%%%%%%%%%%%%%%%%%%%%%%%%%%%%%%%%%%%%%
%%                                          %%
%% Enter the title of your article here     %%
%%                                          %%
%%%%%%%%%%%%%%%%%%%%%%%%%%%%%%%%%%%%%%%%%%%%%%

\title{Relationships between food groups, eating time slots and diabetes status in
  adults from the UK National Diet and Nutrition Survey (2008--2017)}

%%%%%%%%%%%%%%%%%%%%%%%%%%%%%%%%%%%%%%%%%%%%%%
%%                                          %%
%% Enter the authors here                   %%
%%                                          %%
%% Specify information, if available,       %%
%% in the form:                             %%
%%   <key>={<id1>,<id2>}                    %%
%%   <key>=                                 %%
%% Comment or delete the keys which are     %%
%% not used. Repeat \author command as much %%
%% as required.                             %%
%%                                          %%
%%%%%%%%%%%%%%%%%%%%%%%%%%%%%%%%%%%%%%%%%%%%%%

\author[
  addressref={aff1},                   % id's of addresses, e.g. {aff1,aff2}
  % corref={aff1},                       % id of corresponding address, if any
% noteref={n1},                        % id's of article notes, if any
  email={chaochen@wangcc.me}   % email address
]{\inits{C.W.}\fnm{Chaochen} \snm{Wang}}
\author[
  addressref={aff2},
  email={}
]{\inits{S.A.}\fnm{Suzana} \snm{Almoosawi}}
\author[
  addressref={aff3, aff4, aff5},
  corref={aff3},
  email={Luigi.Palla@uniroma1.it}
]{\inits{L.P.}\fnm{Luigi} \snm{Palla}}

%%%%%%%%%%%%%%%%%%%%%%%%%%%%%%%%%%%%%%%%%%%%%%
%%                                          %%
%% Enter the authors' addresses here        %%
%%                                          %%
%% Repeat \address commands as much as      %%
%% required.                                %%
%%                                          %%
%%%%%%%%%%%%%%%%%%%%%%%%%%%%%%%%%%%%%%%%%%%%%%

\address[id=aff1]{%                           % unique id
  \orgdiv{Department of Public Health},             % department, if any
  \orgname{Aichi Medical University},          % university, etc
  \city{Nagakute, Aichi},                              % city
  \cny{Japan}                                    % country
}
\address[id=aff2]{%
  \orgdiv{Faculty of Medicine, School of Public Health},
  \orgname{Imperial College London},
  \city{London},
  \cny{UK}
}
\address[id=aff3]{%
  \orgdiv{Department of Public Health and Infectious Diseases},
  \orgname{University of Rome La Sapienza},
  \street{Piazzale Aldo Moro 5},
  \city{Rome},
  \postcode{00185}
  \cny{Italy}
}
\address[id=aff4]{%
  \orgdiv{Department of Medical Statistics},
  \orgname{London School of Hygiene \& Tropical Medicine},
  \city{London},
  \cny{UK}
}
\address[id=aff5]{%
  \orgdiv{Department of Global Health, School of Tropical Medicine and Global Health},
  \orgname{University of Nagasaki},
  \city{Nagasaki},
  \cny{Japan}
}

%%%%%%%%%%%%%%%%%%%%%%%%%%%%%%%%%%%%%%%%%%%%%%
%%                                          %%
%% Enter short notes here                   %%
%%                                          %%
%% Short notes will be after addresses      %%
%% on first page.                           %%
%%                                          %%
%%%%%%%%%%%%%%%%%%%%%%%%%%%%%%%%%%%%%%%%%%%%%%

%\begin{artnotes}
%%\note{Sample of title note}     % note to the article
%\note[id=n1]{Equal contributor} % note, connected to author
%\end{artnotes}

\end{fmbox}% comment this for two column layout

%%%%%%%%%%%%%%%%%%%%%%%%%%%%%%%%%%%%%%%%%%%%%%%
%%                                           %%
%% The Abstract begins here                  %%
%%                                           %%
%% Please refer to the Instructions for      %%
%% authors on https://www.biomedcentral.com/ %%
%% and include the section headings          %%
%% accordingly for your article type.        %%
%%                                           %%
%%%%%%%%%%%%%%%%%%%%%%%%%%%%%%%%%%%%%%%%%%%%%%%

\begin{abstractbox}

\begin{abstract} % abstract
\parttitle{Background} %if any
Time of eating has been shown to be associated with diabetes and obesity. We aimed to identify potential relationships between foods and eating time, and see whether these associations may vary by diabetes status.
\parttitle{Method} %if any
The National Diet and Nutrition Survey (NDNS) including 6802 adults (age $\geq$ 19 years old) collected 749,026 food recordings by a 4-day-diary. The contingency table cross-classifying 60 food groups with 7 pre-defined eating time slots (6-9am, 9am-12pm, 12-2pm, 2-5pm, 8-10pm, 10pm-6am) were analyzed by Correspondence Analysis (CA). Biplots displaying the associations were generated for all adults and separately by diabetes status (self-reported, pre-diabetes, undiagnosed-diabetes, and non-diabetics) to explore the associations between food groups and time of eating across diabetes strata. For selected food groups, odds ratios (OR, 99\% confidence intervals, CI) were derived of consuming unhealthy foods at evening/night (8pm-6am) vs. earlier in the day, by logistic regression models with generalized estimating equations.
\parttitle{Results} 
The biplots suggested positive associations between evening/night and consumption of puddings, regular soft drinks, sugar confectioneries, chocolates, spirits, beers, ice cream, biscuits, and crisps for all adults in the UK. The OR (99\% CIs) of consuming these foods at evening/night were respectively 1.38 (1.03, 1.86), 1.74 (1.47, 2.06), 1.92 (1.38, 2.69), 3.19 (2.69, 3.79), 11.13 (8.37, 14.80), 7.19 (5.87, 8.82), 2.38 (1.79, 3.15), 1.91 (1.67, 2.16), 1.55 (1.27, 1.88) vs. earlier time. Stratified biplots found that sweetened beverages, sugar-confectioneries appeared more strongly associated with evening/night among un-diagnosed diabetics. 
\parttitle{Conclusions}
Foods consumed in the evening/night time tend to be highly processed, easily accessible, and rich in added sugar or saturated fat. Individuals with undiagnosed diabetes are more likely to consume unhealthy foods at night. Further longitudinal studies are required to ascertain the causal direction of the association between late-eating and diabetes status.
\end{abstract}

%%%%%%%%%%%%%%%%%%%%%%%%%%%%%%%%%%%%%%%%%%%%%%
%%                                          %%
%% The keywords begin here                  %%
%%                                          %%
%% Put each keyword in separate \kwd{}.     %%
%%                                          %%
%%%%%%%%%%%%%%%%%%%%%%%%%%%%%%%%%%%%%%%%%%%%%%

\begin{keyword}
\kwd{chrono-nutrition}
\kwd{time of eating}
\kwd{correspondence analysis}
\kwd{the UK National Diet and Nutrition Survey}
\end{keyword}

% MSC classifications codes, if any
%\begin{keyword}[class=AMS]
%\kwd[Primary ]{}
%\kwd{}
%\kwd[; secondary ]{}
%\end{keyword}

\end{abstractbox}
%
%\end{fmbox}% uncomment this for two column layout

\end{frontmatter}

%%%%%%%%%%%%%%%%%%%%%%%%%%%%%%%%%%%%%%%%%%%%%%%%
%%                                            %%
%% The Main Body begins here                  %%
%%                                            %%
%% Please refer to the instructions for       %%
%% authors on:                                %%
%% https://www.biomedcentral.com/getpublished %%
%% and include the section headings           %%
%% accordingly for your article type.         %%
%%                                            %%
%% See the Results and Discussion section     %%
%% for details on how to create sub-sections  %%
%%                                            %%
%% use \cite{...} to cite references          %%
%%  \cite{koon} and                           %%
%%  \cite{oreg,khar,zvai,xjon,schn,pond}      %%
%%                                            %%
%%%%%%%%%%%%%%%%%%%%%%%%%%%%%%%%%%%%%%%%%%%%%%%%

%%%%%%%%%%%%%%%%%%%%%%%%% start of article main body
% <put your article body there>

%%%%%%%%%%%%%%%%
%% Background %%
%%
\section*{Background}

The timing of energy intake has been shown to be associated with obesity and diabetes. \cite{almoosawi2016chrono} Specifically, eating late at night or having a late dinner was found to be related to higher risk of obesity \cite{xiao2019meal,yoshida2018association}, hyperglycemia \cite{nakajima2015association}, metabolic syndrome \cite{kutsuma2014potential}, diabetes \cite{mattson2014meal}, and poorer glycemic control among diabetics \cite{sakai2017late}. However, the relationship between food choice and the time of food consumption during the day is left largely unknown. Shiftworkers have an increased risk of obesity \cite{balieiro2014nutritional,barbadoro2013rotating}, and diabetes \cite{pan2011rotating}, probably due to limited availability of healthy food choice during their night shifts \cite{bonnell2017influences,balieiro2014nutritional}. Identifying those unhealthy foods that might be chosen during late night time would be helpful when guiding people to change their eating habit for the purpose of either weight losing or maintaining glycemic control. Dietary diary recordings from national surveys can provid detailed food choice data for exploration of the relationships between food groups and their time of consumption in general population.

In this study, we aimed to describe the relationship between food groups and the time of day when they were consumed, and how such relationships may vary by status of type 2 diabetes using the data published by the Rolling Programme of the UK National Diet and Nutrition Survey from 2008 to 2017 as this survey includes diet diaries providing detailed information on the time of day of food intake.

\section*{Methods}

6802 adults (2810 men and 3992 women) and 749026 food recordings collected by the UK National Diet and Nutrition Survey Rolling Programme (NDNS RP 2008-17) were analyzed in the current study \cite{MRCElsieWiddowsonLaboratory2018}. The sample was randomly drawn from a list of all addresses in the UK, clustered into postcode sectors. Details of the rationale, design and methods of the survey can be found in the previous published reports \cite{bates2014national,roberts2018national}. Time of the day was categorized into 7 slots: 6-9 am, 9-12 noon, 12-2 pm, 2-5 pm, 5-8 pm, and 10 pm - 6 am. Foods recorded were classified into 60 standard food groups with 1 to 10 subgroups each: the details are given in Appendix R of the NDNS official report \cite{NDNSdatabase2018}. We focused on the 60 standard food groups in the current analysis. Diabetes status was defined as: 1) healthy if fasting glucose was lower than 6.10 (mmol/L), hemoglobin A1c (HbA1c) were less than 6.5 (\%), and without self-reported diabetes or under treatment of diabetes (n = 2626); 2) pre-diabetic if fasting glucose was lower between 6.10 and 6.99 (inclusive) but without self-reported diabetes or under treatment of diabetes (n = 133); 3) undiagnosed diabetic if either fasting glucose was higher or equal to 7.00 (mmol/L) or HbA1c higher or equal to 6.5 (\%) but without self-reported diabetes or under treatment of diabetes (n = 99); 4) diabetic if participant had self-reported diabetes or under treatment of diabetes (n = 227). Consequently, the number of adults whose diabetes status could not be confirmed was 3717 (1519 men, 2198 women) who were kept only in the whole sample (unstratified) analyses. In addition, the National Statistics Socio-economic Classification \cite{rose2005national} were applied in the survey and therefore accordingly, the socio-economic classification for the individuals' household were defined with 8 categories. 

Correspondence analysis (CA) \cite{greenacre2017correspondence,Chapman2017,palla2020adolescents} was used as a tool for data mining, visualization and hypotheses generation using half of the randomly selected NDNS diary entries data. Specifically, the contingency table generated by cross-tabulating 60 food groups and 7 time slots were analyzed by CA. Through CA, the 60 categories of standard food and the 7 time slots were projected on two dimensions that could jointly contain large percentage of the $\chi^2$ deviation (or inertia) of the table. Biplots that graphically show the association between time of day and food groups were derived for all adults and separately according to their diabetes status. To account for the hierarchical structure of the data (food recorded by the same individuals who lived within the same area/sampling units) and to calculate population average odds ratios (OR), logistic regression models with generalized estimating equations (GEE) were subsequently used to test associations that were first suggested by visual inspection of biplots generated by CA, using the remaining half of the diary entries data. The marginal ORs and their 99\% confidence intervals (CI) were derived of consuming unhealthy food groups selected by CA, later in ther day (8 pm - 6 am, i.e. in the evening and night) compared to earlier in the day (between morning and afternoon). CA and biplots were conducted and generated by the following packages under R environment \cite{Rcoreteam}: \texttt{FactoMineR, factoextra, ggplot2, ggrepel} \cite{L__2008,factoextra,ggplot2,ggrepel}. Logistic regression models with GEE were performed with SAS procedure \texttt{GENMOD} \cite{SAS94} adjusted for age, sex, and socio-economic levels, which were deemed the main potential confounders of the associations. 


\section*{Results}


\section*{Discussion}


\section*{Conclusions}


%Text for this section\ldots
%\subsection*{Sub-heading for section}
%Text for this sub-heading\ldots
%\subsubsection*{Sub-sub heading for section}
%Text for this sub-sub-heading\ldots
%\paragraph*{Sub-sub-sub heading for section}
%Text for this sub-sub-sub-heading\ldots
%
%In this section we examine the growth rate of the mean of $Z_0$, $Z_1$ and $Z_2$. In
%addition, we examine a common modeling assumption and note the
%importance of considering the tails of the extinction time $T_x$ in
%studies of escape dynamics.
%We will first consider the expected resistant population at $vT_x$ for
%some $v>0$, (and temporarily assume $\alpha=0$)
%
%\[
%E \bigl[Z_1(vT_x) \bigr]=
%\int_0^{v\wedge
%1}Z_0(uT_x)
%\exp (\lambda_1)\,du .
%\]
%
%If we assume that sensitive cells follow a deterministic decay
%$Z_0(t)=xe^{\lambda_0 t}$ and approximate their extinction time as
%$T_x\approx-\frac{1}{\lambda_0}\log x$, then we can heuristically
%estimate the expected value as
%%
%\begin{equation}\label{eqexpmuts}
%\begin{aligned}[b]
%&      E\bigl[Z_1(vT_x)\bigr]\\
%&\quad      = \frac{\mu}{r}\log x
%\int_0^{v\wedge1}x^{1-u}x^{({\lambda_1}/{r})(v-u)}\,du .
%\end{aligned}
%\end{equation}
%
%Thus we observe that this expected value is finite for all $v>0$ (also see \cite{koon,xjon,marg,schn,koha,issnic}).





%\section*{Appendix}
%Text for this section\ldots

%%%%%%%%%%%%%%%%%%%%%%%%%%%%%%%%%%%%%%%%%%%%%%
%%                                          %%
%% Backmatter begins here                   %%
%%                                          %%
%%%%%%%%%%%%%%%%%%%%%%%%%%%%%%%%%%%%%%%%%%%%%%

\begin{backmatter}

\section*{Acknowledgements}%% if any
%Text for this section\ldots
The authors sincerely thank ...

\section*{Funding}%% if any
This work was supported by Grants-in-Aid for Young Scientists (grant number 19K20199 to C.W.) from the Japan Society for the Promotion of Science (JSPS).

\section*{Abbreviations}%% if any
to be updated \ldots

\section*{Availability of data and materials}%% if any
Original data used in this study can be accessed upon request to the UK Data Service (https://www.ukdataservice.ac.uk) for academic usage (Study Number: 6533).

\section*{Ethics approval and consent to participate}%% if any
Text for this section\ldots

\section*{Competing interests}
The authors declare that they have no competing interests.

\section*{Consent for publication}%% if any
Text for this section\ldots

\section*{Authors' contributions}
Text for this section \ldots

\section*{Authors' information}%% if any
Text for this section\ldots

%%%%%%%%%%%%%%%%%%%%%%%%%%%%%%%%%%%%%%%%%%%%%%%%%%%%%%%%%%%%%
%%                  The Bibliography                       %%
%%                                                         %%
%%  Bmc_mathpys.bst  will be used to                       %%
%%  create a .BBL file for submission.                     %%
%%  After submission of the .TEX file,                     %%
%%  you will be prompted to submit your .BBL file.         %%
%%                                                         %%
%%                                                         %%
%%  Note that the displayed Bibliography will not          %%
%%  necessarily be rendered by Latex exactly as specified  %%
%%  in the online Instructions for Authors.                %%
%%                                                         %%
%%%%%%%%%%%%%%%%%%%%%%%%%%%%%%%%%%%%%%%%%%%%%%%%%%%%%%%%%%%%%

% if your bibliography is in bibtex format, use those commands:
\bibliographystyle{bmc-mathphys} % Style BST file (bmc-mathphys, vancouver, spbasic).
\bibliography{bmc_article}      % Bibliography file (usually '*.bib' )
% for author-year bibliography (bmc-mathphys or spbasic)
% a) write to bib file (bmc-mathphys only)
% @settings{label, options="nameyear"}
% b) uncomment next line
%\nocite{label}

% or include bibliography directly:
% \begin{thebibliography}
% \bibitem{b1}
% \end{thebibliography}

%%%%%%%%%%%%%%%%%%%%%%%%%%%%%%%%%%%
%%                               %%
%% Figures                       %%
%%                               %%
%% NB: this is for captions and  %%
%% Titles. All graphics must be  %%
%% submitted separately and NOT  %%
%% included in the Tex document  %%
%%                               %%
%%%%%%%%%%%%%%%%%%%%%%%%%%%%%%%%%%%

%%
%% Do not use \listoffigures as most will included as separate files

\section*{Figures}
  \begin{figure}[h!]
  \caption{Sample figure title}
\end{figure}

\begin{figure}[h!]
  \caption{Sample figure title}
\end{figure}

%%%%%%%%%%%%%%%%%%%%%%%%%%%%%%%%%%%
%%                               %%
%% Tables                        %%
%%                               %%
%%%%%%%%%%%%%%%%%%%%%%%%%%%%%%%%%%%

%% Use of \listoftables is discouraged.
%%
\section*{Tables}
\begin{table}[h!]
\caption{Sample table title. This is where the description of the table should go}
  \begin{tabular}{cccc}
    \hline
    & B1  &B2   & B3\\ \hline
    A1 & 0.1 & 0.2 & 0.3\\
    A2 & ... & ..  & .\\
    A3 & ..  & .   & .\\ \hline
  \end{tabular}
\end{table}

%%%%%%%%%%%%%%%%%%%%%%%%%%%%%%%%%%%
%%                               %%
%% Additional Files              %%
%%                               %%
%%%%%%%%%%%%%%%%%%%%%%%%%%%%%%%%%%%

\section*{Additional Files}
  \subsection*{Additional file 1 --- Sample additional file title}
    Additional file descriptions text (including details of how to
    view the file, if it is in a non-standard format or the file extension).  This might
    refer to a multi-page table or a figure.

  \subsection*{Additional file 2 --- Sample additional file title}
    Additional file descriptions text.

\end{backmatter}
\end{document}